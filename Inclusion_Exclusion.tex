\documentclass{article} % Use the article document class

% AMS packages for enhanced math typesetting and symbols:
\usepackage{amsmath}  % Provides enhanced math features like align, gather, etc.
\usepackage{amssymb}  % Provides additional math symbols
\usepackage{amsthm}   % Enables theorem-like environments

% Package for customizing list environments:
\usepackage{enumitem} % Allows control over layout of lists (itemize, enumerate, etc.)

% Full-page layout package:
\usepackage{fullpage} % Uses more of the page area by reducing margins

% TikZ package for drawing graphics:
\usepackage{tikz}     % Used for creating high-quality diagrams and figures

% Microtype package for typographical enhancements:
\usepackage{microtype} % Improves justification, kerning, and overall appearance

% Package for typesetting polynomials:
\usepackage{polynom}  % Provides commands for polynomial long division and related tasks

% Package for controlling figure placement:
\usepackage{placeins} % Provides the \FloatBarrier command to control floating environments

% Forest package for drawing trees:
\usepackage{forest}   % Simplifies the creation of tree diagrams

% Package to allow one LaTeX file to input another:
\usepackage{docmute}  % Allows this file to be included in another document without reloading the preamble

% Load additional TikZ libraries:
\usetikzlibrary{trees} % Provides additional tree-specific commands for TikZ

% Define theorem-like environments using amsthm:
\newtheorem{corollary}{Corollary} % Defines a new "corollary" environment
\newtheorem{lemma}{Lemma}         % Defines a new "lemma" environment


\title{Inclusion-Exclusion Principle}
\author{Tomasz Brengos \\  
Committers : Aliaksei Kudzelka}
\date{}

\begin{document}
\maketitle 
\noindent
\textbf{Very often, we need to calculate the number of elements in the union of certain sets.}
Assuming that we know the sizes of these sets, and their mutual intersections, the principle of
inclusion and exclusion allows us to do exactly that.

\medskip

Suppose you have two sets \(A\) and \(B\). The size of the union is certainly at most \(\lvert A\rvert + \lvert B\rvert\). However, in doing so we count each element of \(A \cap B\) twice. To correct for this, we subtract \(\lvert A \cap B\rvert\) to obtain
\[
|A \cup B| \;=\; |A| + |B| \;-\; |A \cap B|.
\]
In general, the formula gets more complicated because we must take into account intersections of multiple sets. The following statement is what we call the \emph{principle of inclusion and exclusion}:

\begin{lemma}
\label{lem:PIE}
For any collection of finite sets \(A_1, A_2, \ldots, A_n\), we have
\[
\left|\bigcup_{i=1}^{n} A_i\right|
\;=\;
\sum_{\substack{I \subseteq [n] \\ I \neq \emptyset}}
(-1)^{\lvert I\rvert + 1}
\left|\bigcap_{i \in I} A_i\right|.
\]
Equivalently,

\[
\resizebox{\textwidth}{!}{$
|A_1 \cup A_2 \cup \cdots \cup A_n|
\;=\;
\sum_{i=1}^n |A_i|
\;-\;
\sum_{1 \le i < j \le n} |A_i \cap A_j|
\;+\;
\sum_{1 \le i < j < k \le n} |A_i \cap A_j \cap A_k|
\;-\;\cdots\;
+\;
(-1)^{n-1}\,\bigl|A_1 \cap A_2 \cap \cdots \cap A_n\bigr|.
$}
\]


\noindent \textbf{Proof Outline (informal):} Each element that belongs to exactly \(t\) of the sets \(A_i\) is counted \(\binom{t}{1}\) times in the first summation, then subtracted \(\binom{t}{2}\) times in the second summation, added \(\binom{t}{3}\) times in the third, and so on. In other words, its total contribution is
\[
\binom{t}{1} - \binom{t}{2} + \binom{t}{3} - \cdots + (-1)^{t-1}\binom{t}{t},
\]
which equals 1. This alternating sum ensures that each element is ultimately counted exactly once, thereby correcting for any overcounting.

\end{lemma}
\end{document}