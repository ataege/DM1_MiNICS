\documentclass[docmute]{article}
\documentclass{article} % Use the article document class

% AMS packages for enhanced math typesetting and symbols:
\usepackage{amsmath}  % Provides enhanced math features like align, gather, etc.
\usepackage{amssymb}  % Provides additional math symbols
\usepackage{amsthm}   % Enables theorem-like environments

% Package for customizing list environments:
\usepackage{enumitem} % Allows control over layout of lists (itemize, enumerate, etc.)

% Full-page layout package:
\usepackage{fullpage} % Uses more of the page area by reducing margins

% TikZ package for drawing graphics:
\usepackage{tikz}     % Used for creating high-quality diagrams and figures

% Microtype package for typographical enhancements:
\usepackage{microtype} % Improves justification, kerning, and overall appearance

% Package for typesetting polynomials:
\usepackage{polynom}  % Provides commands for polynomial long division and related tasks

% Package for controlling figure placement:
\usepackage{placeins} % Provides the \FloatBarrier command to control floating environments

% Forest package for drawing trees:
\usepackage{forest}   % Simplifies the creation of tree diagrams

% Package to allow one LaTeX file to input another:
\usepackage{docmute}  % Allows this file to be included in another document without reloading the preamble

% Load additional TikZ libraries:
\usetikzlibrary{trees} % Provides additional tree-specific commands for TikZ

% Define theorem-like environments using amsthm:
\newtheorem{corollary}{Corollary} % Defines a new "corollary" environment
\newtheorem{lemma}{Lemma}         % Defines a new "lemma" environment


\title{Counting (Combinatorics)}
\author{Tomasz Brengos \\  
Committers : Mykhailo Moroz}
\begin{document}
\maketitle

\section{Rule of Sum (Addition Principle)}
If a set $S$ is partitioned into disjoint subsets,
\[
S = S_1 \cup S_2 \cup \cdots \cup S_k,
\]
then the total number of elements in $S$ is the sum of the number of elements in each subset:
\[
|S| = |S_1| + |S_2| + \cdots + |S_k|.
\]

\paragraph{Example:}  
Suppose we wish to count the number of ways to choose a subset of a set $X$ of size $u$, but we only consider subsets of a fixed size $k$. If we let $S$ be the family of all such subsets, then using the rule of sum by dividing the choices according to a distinguished element (say, whether a chosen element is included or not) we can count the subsets by summing over the possibilities. (This idea is used later in proofs for binomial coefficients and the power set.)

\paragraph{Theorem:}
\[
\binom{n}{k} = \binom{n-1}{k-1} + \binom{n-1}{k}
\]

\paragraph{Proof:}
Consider \( S = \binom{X}{k} \), the set of all subsets of \( X \) of size \( k \).  
Take any element \( a \in X \). Define:  
 \( S_1 \) as the subsets in \( S \) that contain \( a \) and  
 \( S_2 \) as the subsets in \( S \) that do not contain \( a \).  

Since every subset of \( S \) either contains \( a \) or does not, we see that \( S_1 \) and \( S_2 \) are disjoint and their union forms \( S \), i.e.,  
\[
S_1 \cup S_2 = S.
\]

By the rule of sum, we get:  
\[
|S| = |S_1| + |S_2|.
\]

Now,  
each subset in \( S_1 \) must contain \( a \), so we choose the remaining \( k-1 \) elements from \( X \setminus \{a\} \), which has \( n-1 \) elements. Thus, \( |S_1| = \binom{n-1}{k-1} \).  
Each subset in \( S_2 \) does not contain \( a \), so we choose all \( k \) elements from \( X \setminus \{a\} \). Thus, \( |S_2| = \binom{n-1}{k} \).  

Therefore,  
\[
\binom{n}{k} = |S| = |S_1| + |S_2| = \binom{n-1}{k-1} + \binom{n-1}{k}.
\]

\paragraph{Example:}  
Let \( S = \{ \triangle, \square, \circ \} \) and \( k = 2 \), choosing \( a = \circ \).  
Fixing \( \circ \) as one of the elements in the subset of size \( k \), we get:  
  \[
  S_1 = \{\{\circ, \triangle\}, \{\circ, \square\}\}.
  \]
Taking all subsets of size \( k \) without \( \circ \):  
  \[
  S_2 = \{\{\triangle, \square\}\}.
  \]
We have \( |S_1| = 2 \) and \( |S_2| = 1 \), so \( |S_1| + |S_2| = 3 \).  

\noindent On the other hand,  
\begin{align*}
\binom{3}{2} &= \frac{3!}{2!(3-2)!} = 3.
\end{align*}
Thus, \( |S| = |S_1| + |S_2| = 3 \), verifying the identity.  


\paragraph{}

\section{Rule of Product (Multiplication Principle)}
When an object is constructed by a sequence of choices, where:
\begin{itemize}[nosep]
    \item The first choice can be made in $a$ ways,
    \item The second in $b$ ways,
    \item $\ldots$
\end{itemize}
the total number of objects is the product:
\[
a \times b \times \cdots.
\]

\paragraph{Example:}  
A word of length $n$ over the binary alphabet $\{0,1\}$ is formed by choosing one of $2$ possibilities for each position. Hence, there are
\[
2^n
\]
possible words.

\section{Rule of Bijection}
If there exists a bijection (a one-to-one and onto mapping) between two sets $S$ and $T$, then they have the same number of elements:
\[
|S| = |T|.
\]

\paragraph{Example:}  
Consider the power set of a set $X$, denoted by $\mathcal{P}(X)$. There is a natural bijection between $\mathcal{P}(X)$ and the set of binary sequences of length $|X|$: for each subset $A \subseteq X$, assign the sequence $(a_1,a_2,\dots,a_n)$ where 
\[
a_i = \begin{cases} 
1, & \text{if } x_i \in A, \\
0, & \text{if } x_i \notin A.
\end{cases}
\]
This shows that 
\[
|\mathcal{P}(X)| = 2^{|X|}.
\]

\section{Counting in Two Ways}  

\paragraph{Rule of Counting in Two Ways}  
When two formulae enumerate the same quantity, they must be equal.  

\paragraph{Example:}  
\[
\sum_{i=1}^{n} i = \frac{n(n+1)}{2}
\]

\paragraph{Proof:}  
Consider a lattice grid of size \( (n+1) \times (n+1) \), defined as:  
\[
X = \{(i, j) \mid i, j \in \{1, 2, \dots, n+1\} \}.
\]  
Clearly, \( |X| = (n+1)^2 \).  

Now, partition \( X \) into three subsets:  
- \( X_1 \), the points strictly below the secondary diagonal.  
- \( X_2 \), the points strictly above the secondary diagonal.  
- \( X_3 \), the points on the secondary diagonal itself.  

Since these three sets form a partition, we have:  
\[
|X| = |X_1| + |X_2| + |X_3|.
\]  
Observing their sizes:  
\[
|X_1| = |X_2| = 1 + 2 + \dots + n, \quad |X_3| = n+1.
\]  
Thus,  
\[
(n+1)^2 = 2(1 + 2 + \dots + n) + (n+1).
\]  
Rearranging, we get:  
\[
1 + 2 + \dots + n = \frac{(n+1)^2 - (n+1)}{2} = \frac{n(n+1)}{2}.
\]

Hence, we have proven the formula:  
\[
\sum_{i=1}^{n} i = \frac{n(n+1)}{2}.
\]


\section{Binomial Coefficients and Permutations}
Let $X$ be a set with $|X| = n$.

\paragraph{Subsets:}  
The number of ways to choose a $k$-subset of $X$ is given by the binomial coefficient
\[
\binom{n}{k}.
\]

\paragraph{Permutations:}  
A \( k \)-permutation of a set \( X \) of size \( n \) is a \( k \)-word over the alphabet \( X \) whose entries are distinct.  

\paragraph{Theorem:}  
There are exactly  
\[
n (n-1) (n-2) \dots (n-k+1)
\]  
\( k \)-permutations of an \( n \)-set.  

\paragraph{Question:}  
How are \( k \)-permutations of an \( n \)-set related to \( k \)-subsets of an \( n \)-set?  

\paragraph{Answer:}  
The difference between a \( k \)-permutation and a \( k \)-subset is that a permutation is ordered, while a subset is not.  
To express a \( k \)-permutation in terms of a \( k \)-subset, we need to account for all possible arrangements of the elements, which is \( k! \).  
Thus,  
\[
\text{\( k \)-permutation} = \binom{n}{k} \cdot k!
\]
Expressing \( \binom{n}{k} \) as  
\[
\binom{n}{k} = \frac{n!}{k!(n-k)!}
\]  
we obtain:  
\[
\text{\( k \)-permutation} = \frac{n!}{(n-k)!}.
\]


\paragraph{Proof by Counting in Two Ways:}  
Count the number of $k$-permutations of an $n$-set in two ways:
\begin{enumerate}[label=(\arabic*)]
    \item Directly, by applying the rule of product:
    \[
    n \times (n-1) \times \cdots \times (n-k+1) = \frac{n!}{(n-k)!}.
    \]
    \item First choose a $k$-subset (in $\binom{n}{k}$ ways) and then arrange it (in $k!$ ways), giving
    \[
    \binom{n}{k} \cdot k!.
    \]
\end{enumerate}
Equate these two counts to obtain the relation.

\section{Binomial Theorem}
For any $x,y$ in a field and nonnegative integer $n$, the binomial theorem states:
\[
(x+y)^n = \sum_{k=0}^{n} \binom{n}{k} x^k y^{n-k}.
\]
\paragraph{Explanation:}  
This theorem is a direct consequence of counting the number of ways to choose $k$ copies of $x$ (and the remaining $n-k$ copies of $y$) when expanding the product.

\section{Multisets}  

\paragraph{Definition:}  
A multiset of a set \( X \) of size \( n \) is a function  
\[
m: X \to \mathbb{N}
\]
that assigns a non-negative integer to each element of \( X \), representing its multiplicity in the multiset.  

\paragraph{Example:}  
Let \( X = \{a, b, c\} \), and consider the multiset \( \{a, a, b\} \). Then, the function \( m \) is given by:  
\[
m(a) = 2, \quad m(b) = 1, \quad m(c) = 0.
\]  

\paragraph{Question:}  
What is the number of \( k \)-multisets of a set of size \( n \)?  

\paragraph{Theorem:}  
The number of all \( k \)-multisets of an \( n \)-set is  
\[
\binom{n+k-1}{k}.
\]  

\paragraph{Proof:}  
Let \( X \) be the set of all \( k \)-multisets of an \( n \)-set.  
Let \( Y \) be the set of all distributions of \( k \) identical objects into \( n \) buckets.  

\paragraph{Claim 1:}  
There is a bijection from \( X \) to \( Y \).  
Thus, by the rule of bijection, we have  
\[
|X| = |Y|.
\]  

\paragraph{Claim 2:}  
Let \( Z \) be the set of all binary sequences of length \( n+k-1 \) with exactly \( n-1 \) ones (or equivalently, \( k \) zeros).  
There is a bijection from \( Y \) to \( Z \).  
Hence,  
\[
|Y| = |Z| \Rightarrow |X| = |Z|.
\]  
Since the number of such binary sequences is given by  
\[
\binom{n+k-1}{k},
\]  
we conclude that  
\[
|X| = \binom{n+k-1}{k}.
\]


\section{Lattice Paths}
Consider an $m \times n$ grid with lattice points at the intersections.  
\paragraph{Problem:}  
How many paths are there from $(0,0)$ to $(m,n)$ if one may only move right or up?
\paragraph{Solution:}  
Every path consists of exactly $m$ right moves and $n$ up moves. Thus, a path can be represented as a sequence of $m+n$ moves, where we choose $n$ positions (out of $m+n$) for the up moves. Hence, the number of paths is:
\[
\binom{m+n}{n}.
\]

\paragraph{Bijection Explanation:}  
There is a bijection between the set of such lattice paths and the set of binary sequences of length $m+n$ with exactly $n$ ones (representing the up moves).
\end{document}